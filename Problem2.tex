Let $a$ be an array, $occupied(a)$ be the number of occupied cells in the array, $free(a)$ be the number of free cells in the array.

Assume allocating a new array with $n$ cells takes time $kn$, and deleting an array takes constant time $T_{del}$. We define the potential function of an array $a$ as:

\begin{equation}
\Phi = | occupied(a) - free(a) | * (1 + 2k)
\end{equation}

\section{Amortized performance}

  \subsection{Array Halving}
  It's easy to calculate the potential before and after the operation:
  \[
  \begin{array}{lcl}
    \Phi_{before} & = & | n - 3n | \times (1 + 2k) = 2n \times (1 + 2k) \\
    \Phi_{after} & = & | n - n | \times (1 + 2k) = 0
  \end{array}
  \]


  The real cost of the operation is as follows:
  \[
  C_{real} = n + 2kn + T_{del} + 1
  \]


  Thus, we can derive the amortized cost of array halving:
  \[
  \begin{array}{lcl}
  \hat C & = & C_{real} + \Delta \Phi \\
         & < & 2 C_{real} + \Delta \Phi \\
         & = & 2(n + 2kn + T_{del} + 1) + 0 - 2n \times (1 + 2k) \\
         & = & 2T_{del} + 2 \\
         & = & O(1)
  \end{array}
  \]

  \subsection{Array Doubling}
  The potential before and after the operation are as follows:
  \[
  \begin{array}{lcl}
    \Phi_{before} & = & | n - 0 | \times (1 + 2k) = n \times (1 + 2k) \\
    \Phi_{after} & = & | n - n | \times (1 + 2k) = 0
  \end{array}
  \]

  The real cost of the operation is as follows:
  \[
  C_{real} = n + 2kn + T_{del} + 1
  \]

  Thus, we can derive the amortized cost of array halving:
  \[
  \begin{array}{lcl}
  \hat C & = & C_{real} + \Delta \Phi \\
         & = & n + 2kn + T_{del} + 1 + 0 - n \times (1 + 2k) \\
         & = & T_{del} + 1 \\
         & = & O(1)
  \end{array}
  \]

\section{Halving criterion}
  \textbf{Why not halve the array as soon as the number of elements falls below one half of the allocated size?}
  It can be calculated that $\Delta \Phi = 0$ and $C_{real} = n + nk + T_{del} + 1$. In such case, it's obvious that the amortized cost of this operation is not constant.
