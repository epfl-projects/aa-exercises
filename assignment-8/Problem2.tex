% New ordering through a stack
Let's consider the elements that are before $A[1]$ in the ordered output. There's one important observation:

\begin{quote}
\textit{If $P(1) = k$, then the elements before $A[1]$ are exactly the elements in $A[2], A[3], .., A[k]$.}
\end{quote}

To prove this property, we suppose there exists $A[t]$, such that $t > k$ and $P(t) < P(1)$. According to the push/pop mechanism, $A[t]$ must be pushed onto stack while $A[1]$ is still at the bottom of stack in order to have $P(t) < P(1)$. But we also know that all elements before $A[t]$ must have been pushed to stack previously, as we process the array sequentially. As we know $A[1]$ is always at the bottom of stack when these $t - 1$ elements are pushed onto stack, $A[1]$ must lie behind these $t - 1$ elements in the output, thus we have $P(1) >= t > k$, which is in contradiction with our assumption that $P(1) = k$.

With the above property, we can decouple the problem into the minimal value of a list of subproblems as following:

\[
OPT(A[1..n]) = min \{ OPT(A[1..k]) + k \cdot A[1] + k \cdot \Sigma_{i = k + 1}^{n}A[i] + OPT(A[k+1..n]) \}
\]

In the formula above, we need to add $k \cdot \Sigma_{i = k + 1}^{n}A[i]$ because the rank of each element in the optimal solution of the latter subproblem should increase $k$ in the outer big problem, each generating $k\cdot A[i]$ increase in the total cost.

The algorithm is straight-forward given the recurrence relation above. The algorithm calculates the optimal solutions to all possible intervals $A[i..j]$, starting from the base cases with a single element $A[i]$ and finish at $A[1..n]$.

Now let's see the performance of the algorithm. Obviously, there are exactly $n - i + 1$ intervals with a size of $i$. For each interval of size $i$, it has to select one optimal solution among the $i$ possible divisions of subprolems. So the total cost is $\Sigma_{i=1}^{n} (n - i + 1) \cdot i = O(n^3)$.