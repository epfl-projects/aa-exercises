% Lost dog

\section{Two-way road}
The dog is sitting at a distance $x$ of a bone. Without loss of generality, we assume $x \geq 1$, $1$ being any unit of distance. Since the optimal strategy (\textit{OPT}) knows the exact position of the bone, it only needs to travel a distance $x$.\\

Our strategy is the following :
\begin{enumerate}
  \item Travel a distance $d$ to the right and come back to the central point
  \item Travel a distance $2d$ in the other direction and come back.
  \item Repeat, doubling the distance at each new step and stopping as soon as the bone is found.
\end{enumerate}

Since we assumed $x \geq 1$, we set the initial distance $d = 1$. At round $i$, we travel a distance of $2 \cdot 2^i$ (round trip). We claim that this strategy (\textit{STRAT}) is at least \textit{9-competitive} with \textit{OPT}. The worst-case competitive ratio between \textit{STRAT} and \textit{OPT} occurs in the following situation:\\
When $x = 2^n + \epsilon$, \textit{OPT} only needs to travel a distance $x$ whereas \textit{STRAT} needs to cover an additional distance of:
\begin{itemize}
  \item $2 \cdot 2^n$, coming just $\epsilon$ shy of the bone and coming back to the center (trip number $n$)
  \item $2 \cdot 2^{n+1}$ for the round-trip in the other direction (trip number $n+1$)
  \item $2^n + \epsilon = x$ for finally finding the bone (trip number $n+2$)
\end{itemize}

The total cost of \textit{STRAT} in this worst case amounts to:

\[
\begin{array}{ll}
  C_\text{STRAT} & = \sum_{i=0}^{n+1} {2 \cdot 2^i} + x \\
                 & = 2 \cdot 2^{n+2} + 2^n + \epsilon \\
                 & = 2^n (2^3 + 1) + \epsilon \\
                 & = 9 \cdot 2^n + \epsilon
\end{array}
\]

And the worst-case competitive ratio therefore is at most:
\[
\begin{array}{ll}
  \frac{C_\text{STRAT}}{C_\text{OPT}}
    & = \frac{9 \cdot 2^n + \epsilon}{2^n + \epsilon}\\
    & \leq \frac{9 \cdot (2^n + \epsilon)}{2^n + \epsilon}\\
    & \leq 9\\
\end{array}
\]


\section{\textit{m} roads}

The dog is now sitting at a crossroad with $m$ possible directions. The bone is at a distance $x$ on one of the roads. \textit{OPT} still finds the bone at cost $x$.\\

We cannot apply \textit{STRAT} directly since the cost of a single round would increase exponentially with the number of roads. Instead, we only double the distance after having completed a tour of the $m$ rounds (\textit{STRAT$\prime$}). We claim that \textit{STRAT$\prime$} is at least \textit{$(8m + 1)$-competitive} with \textit{OPT}. If we start by exploring road $1$, the worst case occurs when the bone is at a distance $x = 2^n + \epsilon$ on road $m$. We must then perform another complete tour of the $m-1$ other roads before finding the bone. This final round consists of:
\begin{itemize}
  \item $m-1$ trips of cost $2^{n+1}$ before reaching the correct road
  \item $1$ trip of cost $2^n + \epsilon = x$ for finally finding the bone
\end{itemize}

Similarly to the previous section, the total cost is:

\[
\begin{array}{ll}
  C_{\text{STRAT}\prime}
    & = \sum_{i=0}^n {2m \cdot 2^i} + 2(m-1) \cdot 2^{n+1} + x \\
    & = 2m \cdot \sum_{i=0}^{n+1} {2^i} - (2^{n+1} - x) \\
    & = 2m \cdot 2^{n+2} - 2^{n+1} + 2^n + \epsilon \\
    & \leq 2^n \cdot (2m \cdot 2^2 + 1) - 2^{n+1} + \epsilon \\
    & \leq (2^n + \epsilon) \cdot (8m + 1)
\end{array}
\]

And thus the worst-case competitive ratio is at most $8m + 1$.
