% Alice and Bob play dice
\providecommand{\f}[2]{\ensuremath{\frac{#1}{#2}}}

Alice lets Bob choose first one of the three dices A, B, C. Let us verify that whatever Bob chooses, Alice can choose another dice with which she has a probability of winning $\Pr(W) > {1 \over 2}$.

We denote $A_i$ (resp. $B_i$) the event ``Alice obtains the number $i$'' (resp. Bob). We assume that all dice rolls are independent.

\begin{itemize}
  \item \textbf{Bob chooses dice A}: Alice chooses dice B.\\
  \[
    \begin{array}{ll}
      \Pr(W) & = \Pr(B_1, {A_2\text{ or }A_4\text{ or }A_9}) + \Pr(B_6, A_9) + \Pr(B_8, A_9) \\
             & = \Pr(B_1) \cdot 1 + \Pr(B_6)\Pr(A_9) + \Pr(B_8)(A_9) \\
             & = \f{1}{3} + \f{1}{3} \cdot \f{1}{3} + \f{1}{3} \cdot \f{1}{3} \\
      \Pr(W) & = \f{5}{9} > \f{1}{2}
    \end{array}
  \]

  \item \textbf{Bob chooses dice B}: Alice chooses dice C.\\
  \[
    \begin{array}{ll}
      \Pr(W) & = \Pr(B_2, {A_3\text{ or }A_5\text{ or }A_7}) + \Pr(B_4, {A_5\text{ or }A_7}) \\
             & = \f{1}{3} \cdot 1 + \f{1}{3} \cdot (\f{1}{3} + \f{1}{3}) \\
      \Pr(W) & = \f{5}{9} > \f{1}{2}
    \end{array}
  \]

  \item \textbf{Bob chooses dice C}: Alice chooses dice A.\\
  \[
    \begin{array}{ll}
      \Pr(W) & = \Pr(B_3, {A_6\text{ or }A_8}) + \Pr(B_5, {A_6\text{ or }A_8}) + \Pr(B_7, A_8) \\
             & = \f{1}{3} \cdot (\f{1}{3} + \f{1}{3}) + \f{1}{3} \cdot (\f{1}{3} + \f{1}{3}) + \f{1}{3} \cdot \f{1}{3} \\
      \Pr(W) & = \f{5}{9} > \f{1}{2}
    \end{array}
  \]
\end{itemize}

And thus Alice wins regardless of the choice made by Bob.
