% Quicksort as a randomized incremental construction algorithm

\texttt{Quicksort} is a classical sorting algorithm with worst-case performance in $O(n)$. Let us rephrase \texttt{Quicksort} as a randomized incremental construction algorithm and prove that it achieves $O(n lg(n))$ expected performance.\\

Through the execution, we maintain a \textbf{conflict list} (CL).

The CL can be seen as a bipartite graph representing the conflicts of unprocessed elements with existing intervals. One unprocessed element can only conflict with one interval.
Processing the next element $x$ (chosen at random) corresponds to choosing a new pivot in the original algorithm. We split its original interval into two smaller intervals. This incurs the following cost:
\begin{itemize}
  \item Finding the interval $I$ with which $x$ is in conflict: $O(1)$ thanks to our conflict list.
  \item Removing $x$ from the conflict list (it is no a \textit{processed} element): $O(1)$
  \item Cutting $I$ in half, since $x$ is its new pivot: $O(1)$
  \item Updating the remaining unprocessed elements. Note that only elements which where in conflict with $I$ are affected by the change. They could not enter in conflict with any of the pre-existing intervals, so we only need to decide whether they conflict with one of the new subinterval or the other. This takes time proportional to the number of unprocessed elements in conflict with $I$, which we denote $k$.
\end{itemize}

Let us determine the expected value of $k$ by backwards analysis. At a given step, we have processed $i$ elements, thus creating $i+1$ intervals. Adding the latest element created two of these $i+1$ intervals, with cost proportional to number of elements now in conflict with them. Since elements (pivots) are chosen uniformely at random, the expected number of elements in conflict with each of the intervals is $\frac{n}{i+1}$. Thus step $i$ has expected cost $2 \cdot \frac{n}{i+1}$.

We continue processing elements until all intervals have size 1.
\[
  \sum_{i=0}^n {2 \cdot \frac{n}{i+1}} = 2n \cdot H_{n+1} = O(n lg(n))
\]
With $H_{n+1}$ the $(n+1)^\text{th}$ harmonic number.
