% Produce the number n starting from 1 using two operations (greedy strategy)

\section{The naive greedy strategy does not always produce the optimal solution}

With $n = 10$, we show that the greedy strategy produces a non-optimal solution.

\begin{center}
  \begin{tabular}{|r|c|r|c|}
    \multicolumn{2}{c}{Greedy} & \multicolumn{2}{c}{Optimal} \\
    \hline
    Operation & Result & Operation & Result \\
    \hline
               &  1 &            &  1 \\
    $\times 2$ &  2 & $\times 2$ &  2 \\
    $\times 2$ &  4 & $\times 2$ &  4 \\
    $\times 2$ &  8 & $+1$       &  5 \\
    $\times 2$ &  9 & $\times 2$ & 10 \\
    $+1$       & 10 & \multicolumn{2}{c|}{} \\
    \hline
  \end{tabular}
\end{center}

\section{Another greedy strategy}

Our intuition is that when producing $n$, we should always try to reach intermediate steps where we can use the $\times 2$ operation. Moreover, the larger the number multiplied, the more efficient it is to use the multiplication. This leads us to build the solution in reverse. We propose the following algorithm:

\begin{enumerate}
  \item Start with $i = n$
  \item While $i \neq 1$, repeat:
  \begin{itemize}
    \item If $i$ is even, divide by two (corresponding to $\times 2$ in the solution)
    \item Otherwise, subtract one (corresponding to $+1$ in the solution)
  \end{itemize}
\end{enumerate}

\noindent
The algorithm always makes the optimal choice by using $\times 2$ on the largest possible numbers.
