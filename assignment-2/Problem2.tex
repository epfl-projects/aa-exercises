Let's define a potential function as follows: \\

\[
  \Phi = \sum\limits_{x} \alpha_{x} \times (k - c_{x})
\]

where:
\begin{itemize}
  \item $\alpha_{x}$ is the number of elements before element $x$ in the MTF list but after $x$ in the optimal list (``God’s'' list)
  \item $c_{x}$ is the counter value for $x$
  \item $k$ is the maximum value of that counter $(c_{x} \in [0; k-1])$. When $c_{x} = k$, the element is moved to the front of the list.
\end{itemize}

Say we move some element $x$ (initially at the $i^{th}$ index) to the front of the list on its $k^{th}$ access. Now lets take a look at the cost and how the potential changes:\\
\[
\begin{array}{ll}
  C_{real} & = i\\
  \Delta \Phi & = - \alpha_{x} + \sum\limits_{m} (k - c_m)
\end{array}
\]

Where $m$ belongs to the set of $(i - 1 - \alpha_{x})$ elements that were before $x$ in both the original MTF list and the optimal list. For such elements $\alpha_m$ increases by 1 and hence the potential increase contributed by such element $m$ is $(k - c_m)$. Since, before the move, the counter for $x$ must have been $k - 1$, its potential must have been $\alpha_{x} \times (k - (k - 1)) = \alpha_{x}$. After the move, the potential for $x$ is $0$ since there are no elements before $x$ in the new MTF list. Then we have:

\[
  C_{real} + \Delta \Phi = i - \alpha_{x} + \sum\limits_{m} (k - c_m)
\]

The worst case occurs when for all $m$ elements ($i - 1 - \alpha_{x}$ of them), $c_m$ is $0$ (maximizing potential and hence randomness). In this case, we have:

\[
\begin{array}{ll}
  C_{real} + \Delta \Phi
    & = i - \alpha_{x} + \sum\limits_{m} k \\
    & = i - \alpha_{x} + k(i - 1 - \alpha_{x}) \\
    & = (k+1)(i - 1- \alpha_{x}) + 1
\end{array}
\]

From the regular MTF competitive analysis (seen during the course), we know that the cost of accessing $x$ in the ``God’s list'' is atleast $i - 1 - \alpha_{x}$.\\
So if $j$ is the cost of accessing $x$ in God’s list $j > i - 1 - \alpha_{x}$. Hence,

\[
\begin{array}{ll}
  C_{real} + \Delta \Phi & < (k + 1)j + 1 \\
                         & \leq (k + 1)j
\end{array}
\]

So the amortized cost of moving elements to the front on their $k^{th}$ access is not more than $(k+1)$ times the optimal cost.
