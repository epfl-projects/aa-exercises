% Cover a set of points on the real line (greedy strategy - approximation bound)

\section{Greedy does not always produce the optimal solution}

Consider the following set of points: ${0.1, 0.8, 1.2, 1.9}$. In this example:
\begin{itemize}
  \item The greedy algorithm generates (in order) the following \textbf{three} intervals: $[0.8; 1.8[, [0.1; 1.1[, [1.9, 2.9[$
  \item The optimal solution OPT covers all points with only \textbf{two} intervals: $[0; 1[, [1, 2[$
\end{itemize}

\section{Quality of the greedy solution}

From the previous counter-example, we know that the approximation ratio is bounded below by $\frac{3}{2}$. Let us show that the approximation ratio is also bounded above by $2$.\\

Superimposing the two solutions, we claim that any interval of OPT cannot intersect with more than two intervals of Greedy.

We prove the above claim by contradiction. Suppose three intervals in Greedy solution intersect with an interval in the OPT solution. It implies that the three intervals cover some common range on the axis. Then at least two intervals of the three share a common range more than $\frac{1}{2}$ unit length.\\

Let's analyse how overlapping can occur in the Greedy algorithm. Recall that Greedy only adds intervals where there is a non-covered point. If a newly added interval $m$ intersects with an already added interval $p$, it must also intersect with some other already added interval $q$ ($p$ and $q$ are disjoint). If it's not the case, Greedy can always shift $m$ to be disjoint with $p$ while still covering at least as many points. The gap between $p$ and $q$ must be less than 1, else Greedy can shift $m$ to avoid intersecting with any interval and cover at least as many points. Now we can shift $m$ in such a way that it intersects with $p$ and $q$ over less than $\frac{1}{2}$ unit length. So in the Greedy algorithm two intervals overlap less than $\frac{1}{2}$ unit length. Thus it's impossible for three intervals to cover some common range on the axis. This contradicts our assumption.\\

Therefore, any interval of OPT cannot intersect with more than two intervals of Greedy. If OPT covers all points with $k$ intervals, then Greedy uses at most $2k$ intervals. We obtain an approximation ratio in $[\frac{3}{2}; 2]$ for Greedy. Since we were not able to produce a counter-example in which Greedy perform twice as bad as OPT, it is possible that the approximation ratio is indeed $\frac{3}{2}$.
