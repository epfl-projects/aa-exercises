% Two-steps randomized mincut
\providecommand{\f}[2]{\ensuremath{\cfrac{#1}{#2}}}

\section{Finding nearly optimal values for $k$ and $m$}
  \subsection{Derivation of $\Pr(S_{k, m})$}
  Let $\Pr(S_{k, m})$ denote the probability of success or the probability of not picking min-cut edges in the first k contractions and also during at least one of the $m$ amplification trials. \\

  \[\def\arraystretch{1.5}
    \begin{array}{l l l}
          \Pr(S_{k, m}) & \geq & \prod\limits_{i=0}^{k-1} \frac{n-i-2}{n-i} \times (1 - (1 - \prod\limits_{i=k}^{n-3} \frac{n-i-2}{n-i})^m) \\
               & \geq &  \frac{(n-k)(n-k-1)}{n(n-1)} \times (1 - (1 - \frac{2}{(n-k)(n-k-1)})^m) \\
               & \geq &  \frac{(n-k)(n-k-1)}{n(n-1)} \times (1 - e^{\frac{-2m}{(n-k)(n-k-1)}})
    \end{array}
  \]

  \noindent
  The last substitution is due to the inequality $1-x \leq e^{-x}$.\\
  Using the constraint that the total number of edges contracted by the algorithm is $2(n-2)$, the above inequality can be written as a function of only $n$ and $k$. Since $k+m(n-k-2) = 2(n-2)$, $m = \frac{2(n-2)-k}{(n-k-2)}$ \\
  \[
    \Pr(S_{k, m}) \geq \frac{(n-k)(n-k-1)}{n(n-1)} \times (1 - e^{\frac{-4(n-2)+2k}{(n-k)(n-k-1)(n-k-2)}})
  \]

  \subsection{Optimizing the probability of success}
  In order to maximize this probability one would maximize the right hand side of the above inequality. Let's define the right hand side of the above inquality as a function $f(k)$:
  \[
    f(k) = \frac{(n-k)(n-k-1)}{n(n-1)} \times (1 - e^{\frac{-4(n-2)+2k}{(n-k)(n-k-1)(n-k-2)}})
  \]

  We can compute the optimal value of k by solving $\frac{\partial f(k)}{\partial k} = 0$ or using numerical methods for a given $n$. Once we have the optimal value of $k$ we can get the corresponding value of $m$ by using the equation $m = \frac{2(n-2)-k}{(n-k-2)}$.

\section{Comparison with the fully randomized algorithm}
  We denote $S_{k, m}$ the event ``our algorithm finds the mincut with parameter values $k$ and $m$''. We denote $\Pr(S_{k, m})$ the associated probability, and $\Pr_0(S_m)$ the probability of success of the original, fully randomized algorithm over $m$ runs.

  \noindent
  As derived in the lecture notes, the original algorithm achieves the following performance:
  \[
    {\Pr}_0(S_m) = (1 - \f{2}{n(n-1)})^m
  \]

  \noindent
  In our case, we fixed the reference to $m = 2$, so our algorithm must compete with $\Pr_0(S_2) = (1 - \f{2}{n(n-1)})^2$.

