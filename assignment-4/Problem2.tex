% Separate red and green points with a line (2D linear programming)

Suppose the $n$ red points are $(x_1^r, y_1^r), (x_2^r, y_2^r), .., (x_n^r, y_n^r)$. The $m$ green points are $(x_1^g, y_1^g), (x_2^g, y_2^g), .., (x_m^g, y_m^g)$. Suppose the line $l$ which separates the red points from green points is as follows:

\[
  a \cdot x + b \cdot y + 1 = 0
\]

Note that the assumption excludes sparating lines like $a \cdot x + b \cdot y = 0$. However, if $a \cdot x + b \cdot y = 0$ is a separating line, we can move the line slightly to get a new separating line like $a \cdot x + b \cdot y + \epsilon = 0$, which is equal to $\frac{a}{\epsilon} \cdot x + \frac{b}{\epsilon} \cdot y + 1 = 0$.

As line $l$ halves the plane, the $n$ red points and $m$ green points are on different side of the line, we have:

\[
  \left\{
      \begin{array}{l}
          a \cdot x_1^r + b \cdot y_1^r + 1 < 0 \\
          a \cdot x_2^r + b \cdot y_2^r + 1 < 0 \\
          \cdots \\
          a \cdot x_n^r + b \cdot y_n^r + 1 < 0 \\
          a \cdot x_1^g + b \cdot y_1^g + 1 > 0 \\
          a \cdot x_2^g + b \cdot y_2^g + 1 > 0 \\
          \cdots \\
          a \cdot x_m^g + b \cdot y_m^g + 1 > 0
      \end{array}
  \right.
  \hspace{1cm}\text{or}\hspace{1cm}
  \left\{
      \begin{array}{l}
          a \cdot x_1^r + b \cdot y_1^r + 1 > 0 \\
          a \cdot x_2^r + b \cdot y_2^r + 1 > 0 \\
          \cdots \\
          a \cdot x_n^r + b \cdot y_n^r + 1 > 0 \\
          a \cdot x_1^g + b \cdot y_1^g + 1 < 0 \\
          a \cdot x_2^g + b \cdot y_2^g + 1 < 0 \\
          \cdots \\
          a \cdot x_m^g + b \cdot y_m^g + 1 < 0
      \end{array}
  \right.
\]

We have two set of inequalities above, due to the fact that we don't know in advance on which side of the line the red and green points lie respectively.

It's obvious that the inequalities above are linear constraints about variable $a$ and $b$. If we define the objective function as follows:

\[
  f(a, b) = a + b
\]

Then this is a pure 2D linear programming problem, we just need to run the standard algorithm twice. As the expected cost of the standard randomized algorithm is \~O(n) for n constraints, so we can find the separating line in \~O(m + n).