\section{Example usage of an infinite stack}
Starting with an empty infinite stack $S = S_0, S_1, S_2, \ldots$, we \texttt{push} the 15 elements $1, 2, 3,\ldots, 15$. We obtain the following configuration (top of the stack to the right):

\[
\begin{array}{lcccccccccc}
  S_0 & = & 15\\
  S_1 & = & 13 & 14\\
  S_2 & = & 10 & 9 & 12 & 11\\
  S_3 & = & 3 & 4 & 1 & 2 & 7 & 8 & 5 & 6
\end{array}
\]

We then perform 7 \texttt{pop} operations and obtain:

\[
\begin{array}{lcccccccccc}
  S_0 & = & 8\\
  S_1 & = & 7\\
  S_2 & = & 6 & 5 &\\
  S_3 & = & 3 & 4 & 1 & 2
\end{array}
\]

\section{Worst-case complexity}
Let $n$ be the current number of elements in an infinite stack $S$.

  \subsection{\texttt{push}}
  The worst-case for insertion occurs when all stacks $S_0\ldots S_m$ are full, i.e. each stack $S_i$ contains $2^i$ elements. Pushing thus triggers an overflow of all stacks. The cost of an overflow of stack $S_i$ is $2^i$ \texttt{pop} operations and $2^i$ \texttt{push} operations on the next stack $S_{i+1}$.\\
  For this configuration to happen, we must have:

  \[
  \begin{array}{lccl}
    & n & = & \sum_{i=0}^m 2^i = 2^m - 1\\
    \iff & m & = & log_2(n + 1)
  \end{array}
  \]

  Thus the worst-case cost of \texttt{push} amounts to:

  \[
  \begin{array}{ll}
    C_{push} & = \sum_{i=0}^m k_{push} 2^i + \sum_{i=0}^m k_{pop} 2^i + 1\\
             & \leq 2 \text{max}(k_{push}, k_{pop}) * \sum_{i=0}^m 2^i\\
             & = 2 k_{\text{max}} n\\
             & = O(n)
  \end{array}
  \]


\section{\texttt{push} and \texttt{pop} cannot be amortized to constant time}
