\section{Example usage of an infinite stack}
Starting with an empty infinite stack $S = S_0, S_1, S_2, \ldots$, we \texttt{push} the 15 elements $1, 2, 3,\ldots, 15$. We obtain the following configuration (top of the stack to the right):

\[
\begin{array}{lcccccccccc}
  S_0 & = & 15\\
  S_1 & = & 13 & 14\\
  S_2 & = & 10 & 9 & 12 & 11\\
  S_3 & = & 3 & 4 & 1 & 2 & 7 & 8 & 5 & 6
\end{array}
\]

We then perform 7 \texttt{pop} operations and obtain:

\[
\begin{array}{lcccccccccc}
  S_0 & = & 8\\
  S_1 & = & 7\\
  S_2 & = & 6 & 5 &\\
  S_3 & = & 3 & 4 & 1 & 2
\end{array}
\]

\section{Worst-case complexity}
Let $n$ be the current number of elements in an infinite stack $S$.

  \subsection{\texttt{push}}
  \label{worst-case-push}
  The worst-case for insertion occurs when all stacks $S_0\ldots S_m$ are full, i.e. each stack $S_i$ contains $2^i$ elements. Pushing thus triggers an overflow of all stacks. The cost of an overflow of stack $S_i$ is $2^i$ \texttt{pop} operations and $2^i$ \texttt{push} operations to the next stack $S_{i+1}$.\\
  For this configuration to happen, we must have:

  \[
    n = \sum_{i=0}^m 2^i = 2^m - 1
  \]

  The worst-case cost of \texttt{push} amounts to:

  \[
  \begin{array}{ll}
    C_{push} & = \sum_{i=0}^m k_{push} 2^i + \sum_{i=0}^m k_{pop} 2^i + 1\\
             & \leq 2 * \text{max}(k_{push}, k_{pop}) * \sum_{i=0}^m 2^i + 1\\
             & = 2 k_{\text{max}} n + 1\\
             & = O(n)
  \end{array}
  \]

  \subsection{\texttt{pop}}
  Similarly, the worst-case for deletion occurs when stacks $S_0$ has one element and stacks $S_1\ldots S_m$ each contains $2^{i-1}$ elements. Poping leads $S_0$ to being empty and refill from $S_1$, which in turn becomes empty and must refill from $S_2$, and so on. The cost of refilling stack $S_i$ from stack $S_{i+1}$ is $2^i$ \texttt{pop} operations and $2^i$ \texttt{push} operations.

  The worst-case cost of \texttt{push} amounts to:

  \[
  \begin{array}{ll}
    C_{pop} & = \sum_{i=0}^m k_{push} 2^{i-1} + \sum_{i=0}^m k_{pop} 2^{i-1} + 1\\
            & \leq \text{max}(k_{push}, k_{pop}) * \sum_{i=0}^m 2^i + 1\\
            & = k_{\text{max}} n + 1\\
            & = O(n)
  \end{array}
  \]

\section{\texttt{push} and \texttt{pop} cannot be amortized to constant time}

Let us show that the operations \texttt{push} and \texttt{pop} of the infinite stack cannot be amortized to constant time. Let $\Phi$ any potential function. If the operations could be amortized to $O(1)$, then for all sequences of operations, we would have: $\exists \alpha$ such that

\[
\begin{array}{lrcl}
  & \sum C_{real} + \sum \Delta\Phi & \leq \sum \alpha \\
  \iff & C_{sequence} + \Phi_{final} - \Phi_{initial} & \leq \alpha n
\end{array}
\]

Let $S$ an infinite stack containing $n$ elements in the initial configuration the worst-case described in section \ref{worst-case-push}.\\
We perform the following sequence of operations: \texttt{push} followed by \texttt{pop}, repeated $n$ times. As described earlier, the \texttt{push} operations trigger overflows amounting to total cost $O(n)$ and the \texttt{pop} operations trigger refills amounting to total cost $O(n)$. Since we perform $n$ such operations, we obtain $C_{sequence} = O(n^2)$. The complete sequence returns the data structure in its initial state, so that $\Delta\Phi_{sequence} = 0$.\\
Finally, we obtain:

\[
\begin{array}{rl}
  C_{sequence} + \Phi_{final} - \Phi_{initial} & = O(n^2) + 0 \\
   & \geq \alpha n, \forall \alpha
\end{array}
\]

Which shows that the operations cannot be amortized to constant time regardless of the potential function chosen.
