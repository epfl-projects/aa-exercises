\documentclass[a4paper,11pt]{article}
%
% HOW TO USE THIS TEMPLATE:
%
% IN THE TWO INDICATED PLACES BELOW,
%   ENTER YOUR NAMES AND THE NUMBER OF THE HOMEWORK
%
% FOR EACH PROBLEM IN THE SET, WRITE YOUR SOLUTION IN A FILE
%   NAMED "Problem1.tex", "Problem2.tex", etc.
%   THESE SOLUTIONS FILES HAVE NO PREAMBLE AND
%   NO \begin{document} \end{document}
%   (THEY JUST GET INLINED WHEN RUNNING LaTeX)
%
% PLACE THESE SOLUTIONS FILES AND THE TEMPLATE IN THE SAME DIRECTORY
%
% RUN LaTeX
%
% IF YOU NEED TO ADD PACKAGES (E.G., FOR HANDLING FIGURES, OR
%   FOR RUNNING PDFLaTeX), PUT IN A NEW \usepackage LINE AS
%   SHOWN BELOW

%
%%%% REPLACE Doe, Smith, and Wilson WITH YOUR LAST NAMES
%%%% NOTE: IF YOUR LAST NAME HAS ACCENTS, SOME LaTeX COMMANDS ARE:
%%%%       for diaeresis/umlaut ", as in German, use \"a, \"u, \"o, etc.
%%%%       for grave accent `, use \`e, \`a, \`u
%%%%       for acute accent ', use \'a, \'e, \'u
%%%%       for tilde ~, use \tilde{a}, \tilde{o}
%%%%       for caret ^, use \hat{e}, \hat{o}
%%%%	   if you need an accent on an i, say an umlaut, use \"\i{}
%%%%		the \i{} is an i without its dot; e.g., na\"\i{}ve
%%%%	   these all work well with lower-case letters,
%%%%	        but only so-so with upper-case letters
%%%%
  \def\myname{Fengyun Liu, Nimier-David, and Sapkota}

%%%% REPLACE xxx WITH CORRECT HOMEWORK NUMBER
%  \def\mynumber{xxx}
  \def\mynumber{6}

  \newcounter{problems}
%%%% REPLACE yyy WITH CORRECT NUMBER OF PROBLEMS IN HOMEWORK
%%%% yyy SHOULD BE A NUMBER BETWEEN 4 AND 7
%  \setcounter{problems}{yyy}
  \setcounter{problems}{4}

%%%% IF ADDITIONAL PACKAGES ARE NEEDED, UNCOMMENT LINE BELOW
%%%% AND ENTER THE PACKAGE NAMES
%  \usepackage{additional_package_1,additional_package_2}
\usepackage{graphicx}

% DO NOT MODIFY THE REST OF THE FILE
%%%% Recursive functions
  \def\tableformat#1{
    |c
    \ifnum#1<\theproblems
      \tableformat{\number\numexpr#1+1}
    \fi
  }

  \def\tablecontent#1{
    \ifnum#1<\theproblems
      Prob.~#1 &
      \tablecontent{\number\numexpr#1+1}
    \fi
    \ifnum#1=\theproblems
      Prob.~#1
    \fi
  }

  \def\tablevoid#1{
    \ifnum#1<\theproblems
      &
      \tablevoid{\number\numexpr#1+1}
    \fi
  }

  \def\problemscontent#1{
    \ifnum#1>\theproblems\else
      \ifnum#1>1\newpage\fi

      \addtocounter{problem}{1}
      \setcounter{section}{0}

      \begin{flushleft}
        \large\sf Problem \theproblem .
      \end{flushleft}
      \input{Problem#1.tex}
      \problemscontent{\number\numexpr#1+1}
    \fi
  }

%%%%%
  \usepackage{times,amsmath,pslatex,graphicx,ifthen}
  \newcounter{problem}
  \setcounter{problem}{0}
  \makeatletter
  \def\ps@headings{%
    \let\@mkboth\markboth
    \def\@evenfoot{\hfil\large\sf Homework \mynumber, Problem \theproblem\hfil}
    \def\@oddfoot{\@evenfoot}
    \def\@evenhead{\large\sc\myname\hfil\it\today\hfil\sf P.~\thepage}
    \def\@oddhead{\@evenhead}}
  \makeatother
  \pagestyle{headings}
  \advance\textwidth by16mm
  \advance\oddsidemargin by-8mm
  \advance\textheight by10mm
%%%%%

\begin{document}

\begin{center}
  \LARGE\sf
  \begin{tabular}{|\tableformat{1}||}
    \hline\hline
    \tablecontent{1} \\
    \hline
    \tablevoid{1}\\
    \hline\hline
  \end{tabular}
\end{center}

\bigskip\bigskip

\problemscontent{1}

\end{document}
